\documentclass[a4paper]{article}
\usepackage[margin=1.5cm]{geometry}
\usepackage{amsmath}
\usepackage{amssymb}
\usepackage{enumitem}
\usepackage{tikz}
\usepackage{url}
\usetikzlibrary{matrix}
\begin{document}
\pagestyle{empty}

\begin{enumerate}[itemsep=1em,leftmargin=*]
  \item Reduksi Dimensi dan Optimasi
  
  ABSORB:

  Cermati materi tentang reduksi dimensi berikut ini :
  \begin{enumerate}
    \item \url{https://www.geeksforgeeks.org/machine-learning/dimensionality-reduction}
    \item \url{https://dqlab.id/mengenal-dimensionality-reduction-dalam-machine-learning}
  \end{enumerate}

  DO:

  Dalam konteks optimisasi model machine learning, apakah reduksi dimensi (contohnya menggunakan PCA atau Autoencoder) selalu menguntungkan?
  
  Dukung pendapatmu dengan menggunakan contoh.

  Reduksi dimensi dalam konteks optimisasi model machine learning tidak selalu menguntungkan secara mutlak, meskipun sering membawa banyak manfaat. Teknik reduksi dimensi seperti PCA (Principal Component Analysis) atau Autoencoder memiliki keuntungan signifikan dalam mempercepat pelatihan model, mengurangi kompleksitas data, serta mencegah overfitting dengan menghilangkan fitur yang tidak relevan atau redundan. Namun, ada risiko kehilangan informasi penting yang dapat menurunkan akurasi model jika fitur utama yang relevan ikut terbuang.

  Keuntungan Reduksi Dimensi

  \begin{enumerate}
    \item Mempercepat proses training dan inferensi model karena jumlah fitur yang lebih sedikit sehingga komputasi lebih efisien.
    \item Mencegah overfitting dengan menghilangkan fitur yang tidak relevan, sehingga model lebih generalisasi pada data baru.
    \item Membuat data lebih mudah diinterpretasi dan divisualisasikan, khususnya pada data berdimensi tinggi.
    \item Autoencoder bahkan dapat mempelajari representasi non-linear fitur data, yang bisa lebih powerful daripada PCA yang linear.
  \end{enumerate}

  Kerugian dan Risiko

  \begin{enumerate}
    \item Reduksi dimensi dapat menyebabkan kehilangan informasi yang berharga untuk prediksi jika fitur penting ikut terbuang.
    \item PCA yang hanya linear kadang kurang efektif jika fitur data memiliki hubungan non-linear, sehingga potensi informasi bisa hilang.
    \item Autoencoder yang kompleks memerlukan tuning dan arsitektur yang tepat agar tidak menimbulkan underfitting atau overfitting pada representasi latennya.
  \end{enumerate}

  Contoh Kasus

  Misalnya pada dataset citra wajah dengan ribuan piksel sebagai fitur, Autoencoder digunakan untuk mereduksi dimensi menjadi representasi latent yang jauh lebih kecil (misal dari ribuan menjadi puluhan dimensi). Representasi ini masih cukup kaya informasi untuk klasifikasi wajah (misal beda jenis kelamin) dengan akurasi yang baik dan waktu pelatihan lebih cepat dibanding langsung menggunakan data mentah. Namun, jika terlalu banyak dimensi direduksi atau arsitektur autoencoder kurang tepat, akurasi model bisa menurun.
  
  Jadi, apakah reduksi dimensi selalu menguntungkan? Jawabannya tergantung pada konteks dan cara penerapannya. Reduksi dimensi yang tepat dapat sangat menguntungkan dengan membuat model lebih efisien dan general, namun jika tidak hati-hati, bisa terjadi kehilangan informasi penting yang merugikan performa model.
  
  Referensi mendalam: teknologi PCA untuk linear dan autoencoder untuk non-linear, dampak efisiensi dan risiko loss informasi, serta contoh pengaplikasian pada dataset high-dimensional.

  Referensi:

  \begin{enumerate}
    \item \url{https://www.perplexity.ai/search/dalam-konteks-optimisasi-model-eCgeOEhxQPCqP3mBo5Mo8A#0}
  \end{enumerate}
  
  \item Penggunaan Calculus di Machine Learning
  
  DO:

  Bagaimana konsep turunan (derivatif / diferensial) dalam kalkulus, khususnya dalam metode seperti gradient descent, memainkan peran krusial dalam proses 'belajar' atau optimisasi parameter pada model machine learning seperti regresi linier atau jaringan saraf tiruan?

  Konsep turunan (derivatif atau diferensial) dalam kalkulus adalah pengukuran bagaimana suatu fungsi berubah seiring perubahan inputnya, atau secara informal, seberapa cepat suatu nilai berubah terhadap perubahan variabelnya. Dalam konteks machine learning, terutama pada metode seperti gradient descent yang digunakan di regresi linier maupun neural network, konsep turunan ini berperan fundamental dalam proses "belajar" atau optimisasi parameter model.

  Arti Turunan dalam Kalkulus

  Turunan dari suatu fungsi $f(x)$, dilambangkan sebagai $f'(x)$ atau $\frac{df}{dx}$, menunjukkan laju perubahan $f(x)$ terhadap $x$. Secara geometris, ini adalah gradien (kemiringan) garis singgung pada grafik fungsi di titik tertentu. Misalnya, jika $f(x) = x^2$, maka $f'(x) = 2x$, yang menunjukkan bahwa untuk setiap perubahan kecil pada $x$, perubahan pada $f(x)$ sekitar dua kali perubahan $x$.

  Peran Turunan dalam Gradient Descent

  Dalam machine learning, model seperti regresi linier atau jaringan saraf tiruan memiliki parameter (misal, bobot dan bias) yang harus dioptimasi agar model memberikan prediksi terbaik. Proses optimasi ini sering kali menggunakan algoritma gradient descent, yang bertujuan meminimalkan fungsi loss (kerugian) dengan memodifikasi parameter secara bertahap.

  \begin{enumerate}
    \item Turunan dari fungsi loss terhadap tiap parameter menunjukkan arah dan seberapa besar perubahan parameter tersebut agar loss berkurang.
    \item Gradient descent menggunakan informasi gradien (turunan) ini untuk memperbarui parameter secara iteratif ke arah nilai minimum loss.
    \item Jika gradiennya positif, parameter dikurangi; jika gradien negatif, parameter ditambah, sesuai prinsip menuruni lereng menuju titik terendah (minimum lokal atau global).
  \end{enumerate}

  Pengaruh pada Proses Belajar Model

  \begin{enumerate}
    \item Setiap iterasi training, model menghitung turunan dari fungsi loss terhadap setiap parameter—proses ini disebut backpropagation di neural network.
    \item Turunan memungkinkan model untuk secara sistematis menemukan kombinasi parameter terbaik yang meminimalkan error prediksi.
  \end{enumerate}

  Ringkasan

  Turunan adalah inti matematis dari proses optimasi model-machine learning. Tanpa turunan, algoritma seperti gradient descent tidak bisa mengetahui arah dan besar langkah untuk memperbaiki parameter model secara efisien, sehingga proses "belajar" tidak akan terjadi optimal.

  \begin{enumerate}
    \item \url{https://www.perplexity.ai/search/bagaimana-konsep-turunan-deriv-DGqg3sZrSNqEd7zMfSVtJg#0}
  \end{enumerate}

  \item Regresi pada Machine Learning
  
  ABSORB:

  Cermati materi tentang penggunaan regresi pada pembelajaran mesin berikut:

  \begin{enumerate}
    \item \url{https://www.geeksforgeeks.org/machine-learning/ml-linear-regression}
  \end{enumerate}

  dan pelajari regresi dari video berikut :

  \begin{enumerate}
    \item \url{https://www.youtube.com/watch?v=qxo8p8PtFeA}
  \end{enumerate}

  DO:

  Seorang peneliti ingin mengetahui apakah ada hubungan linear antara lama waktu belajar (X, dalam jam) per hari dengan nilai ujian (Y) yang diperoleh siswa.
  
  Peneliti tersebut mengumpulkan data dari 6 siswa sebagai berikut:

  \begin{table}[h!]
  \centering
  \begin{tabular}{|c|c|c|c|c|c|c|}
  \hline
  \textbf{Siswa} & 1 & 2 & 3 & 4 & 5 & 6 \\ \hline
  \textbf{Lama belajar (X)} & 1 & 2 & 3 & 4 & 5 & 6 \\ \hline
  \textbf{Nilai Ujian (Y)} & 60 & 65 & 75 & 80 & 85 & 90 \\ \hline
  \end{tabular}
  \end{table}

  Tentukan persamaan garis regresi linear yang memodelkan hubungan antara lama belajar dan nilai ujian tersebut.

  Diketahui:

  \(n = 6\)

  \(\sum X = 1 + 2 + 3 + 4 + 5 + 6 = 21\)

  \(\sum Y = 60 + 65 + 75 + 80 + 85 + 90 = 455\)

  \(\sum X^2 = 1^2 + 2^2 + 3^2 + 4^2 + 5^2 + 6^2 = 91\)

  \(\sum XY = (1)(60) + (2)(65) + (3)(75) + (4)(80) + (5)(85) + (6)(90) = 1700\)

  Dengan rumus garis regresi:

  \(\quad Y = a + bX\)

  \(b = \frac{n\sum XY - (\sum X)(\sum Y)}{n\sum X^2 - (\sum X)^2}\)

  \(a = \frac{\sum Y - b\sum X}{n}\)

  Mencari nilai \(a\) dan \(b\) :

  \(a = \frac{455 - (6.142857)(21)}{6}\)

  \(a = 54.333333\) atau \(a = \frac{163}{3}\)

  \(b = \frac{6(1700) - (21)(455)}{6(91) - (21)^2}\)

  \(b = \frac{10200 - 9555}{546 - 441}\)

  \(b = \frac{645}{105}\)

  \(b = 6.142857\) atau \(b = \frac{43}{7}\)

  Jadi

  \(\hat{Y} = 54.33 + 6.14X\) atau \(\hat{Y} = \frac{163}{3} + \frac{43}{7}X\)

\end{enumerate}
\end{document}
