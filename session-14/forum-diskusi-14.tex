\documentclass[a4paper]{article}
\usepackage[margin=1.5cm]{geometry}
\usepackage{amsmath}
\usepackage{amssymb}
\usepackage{enumitem}
\usepackage{tikz}
\usepackage{url}
\usetikzlibrary{matrix}
\begin{document}
\pagestyle{empty}

\begin{enumerate}[itemsep=1em,leftmargin=*]
  \item Interpolasi Invers
  
  ABSORB:

  Cermati materi tentang interpolasi invers pada materi-materi berikut:
  \begin{enumerate}
    \item \url{https://www.geeksforgeeks.org/dsa/program-to-implement-inverse-interpolation-using-lagrange-formula/}
    \item \url{https://www.sangakoo.com/en/unit/inverse-interpolation}
  \end{enumerate}

  DO:

  Interpolasi invers memiliki satu kelemahan yakni ia bekerja dengan baik jika fungsi tersebut mononik pada interval yang ditinjau.

  \begin{enumerate}
    \item Apa arti monotonik dalam hal ini?
    
    Dalam konteks interpolasi invers, kata monotonik mengacu pada sifat fungsi yang selalu bergerak satu arah saja — selalu naik atau selalu turun — tanpa berbalik arah di dalam interval yang sedang dibahas.

    Secara matematis:

    \begin{enumerate}
      \item Fungsi dikatakan \textbf{monoton naik} (monotonik naik) jika
      
      \(x_1 < x_2 \Rightarrow f(x_1) \le f(x_2)\)

      Artinya, ketika \(x\) bertambah, nilai \(f(x)\) juga bertambah (atau tetap).

      \item Fungsi dikatakan \textbf{monoton turun} (monotonik turun) jika
      
      \(x_1 < x_2 \Rightarrow f(x_1) \ge f(x_2)\)

      Artinya, ketika \(x\) bertambah, nilai \(f(x)\) berkurang (atau tetap).
    \end{enumerate}
    
    \textbf{Jawaban:}
    
    \item Mengapa monotonik menjadi masalah dalam hal interpolasi invers?
    
    \textbf{Jawaban:}

    Jika fungsi tidak monotonik, maka satu nilai  \(y\) bisa berasal dari lebih dari satu nilai \(x\).

    Akibatnya:

    \begin{enumerate}
      \item Fungsi invers \(f^{-1}(y)\) tidak terdefinisi dengan baik karena tidak ada cara unik untuk menentukan nilai \(x\) dari nilai \(y\).
      \item Interpolasi invers dapat menghasilkan hasil yang ambigu atau tidak akurat karena tidak ada satu solusi tunggal untuk nilai \(x\) yang sesuai dengan nilai \(y\) tertentu.
      \item Hasil interpolasi bisa sangat bergantung pada titik data yang dipilih, sehingga sulit untuk mendapatkan hasil yang konsisten.
    \end{enumerate}

  \end{enumerate}

  \item Polinomial Interpolasi
  
  ABSORB:

  Cermati materi tentang interpolasi pada Lecturer Notes week 9 dengan didukung beberapa materi berikut:

  \begin{enumerate}
    \item \url{https://www.geeksforgeeks.org/maths/interpolation-formula/}
    \item \url{https://www.cuemath.com/linear-interpolation-formula/}
  \end{enumerate}

  DO:

  Kita akan mengukur suhu :

  \begin{itemize}
    \item Pada jam 2 siang \((X_0 = 2)\), suhunya 20 derajat \((Y_0 = 20)\).
    \item Pada jam 6 sore \((X_1 = 6)\), suhunya 12 derajat \((Y_1 = 12)\).
  \end{itemize}

  Tentukan suhu pada jam 4 sore \((X = 4)\).

  \textbf{Jawaban:}

  \(X_0 = 2, Y_0 = 20, X_1 = 6, Y_1 = 12, X = 4\)

  Rumus interpolasi linear:

  \(Y = Y_0 + (X - X_0)\frac{(Y_1 - Y_0)}{(X_1 - X_0)}\)

  Langsung substitusikan nilai yang diketahui:

  \(Y = 20 + (4 - 2)\frac{(12 - 20)}{(6 - 2)}\)

  \(Y = 20 + 2 \times \frac{-8}{4}\)

  \(Y = 20 - 4\)

  \(Y = 16\)

  \item Interpolasi Langrange
  
  ABSORB:

  Cermati materi tentang interpolasi Langrange terkait dengan kalkulus yang terdapat pada materi-materi berikut ini :

  \begin{enumerate}
    \item \url{https://www.geeksforgeeks.org/maths/lagrange-interpolation-formula/}
    \item \url{https://mathworld.wolfram.com/LagrangeInterpolatingPolynomial.html}
    \item \url{https://brilliant.org/wiki/lagrange-interpolation/}
  \end{enumerate}

  DO:

  Sebuah sensor mengikur data pada titik-titik berikut : \((1, 2)\); \((2, 3)\); dan \((4, 11)\).

  \begin{itemize}
    \item Tentukan polinomial interpolasi langrange yang melewati ketiga titik tersebut.
    \item Gunakan polinomial yang anda temukan untuk mengestimasi nilai \(Y\) ketika \(X = 3\).
  \end{itemize}

  \textbf{Jawaban:}

  Diketahui titik-titik :
  
  \((x_0, y_0) = (1, 2)\); \((x_1, y_1) = (2, 3)\); dan \((x_2, y_2) = (4, 11)\).

  Untuk polinomial interpolasi Lagrange, rumusnya adalah:

  \(P(x) = \sum_{i=0}^{n} y_i L_i(x)\)

  Sehingga untuk titik-titik di atas didapatkan:

  \(P(x) = y_0 L_0(x) + y_1 L_1(x) + y_2 L_2(x)\)

  \(L_0(x) = \frac{(x - 2)(x - 4)}{(1 - 2)(1 - 4)} = \frac{(x - 2)(x - 4)}{3}\)

  \(L_1(x) = \frac{(x - 1)(x - 4)}{(2 - 1)(2 - 4)} = -\frac{(x - 1)(x - 4)}{2}\)

  \(L_2(x) = \frac{(x - 1)(x - 2)}{(4 - 1)(4 - 2)} = \frac{(x - 1)(x - 2)}{6}\)

  Polinomial lengkap didapatkan dengan substitusi nilai \(y_i\):

  \(P(x) = 2 \cdot \frac{(x - 2)(x - 4)}{3} + 3 \cdot \left(-\frac{(x - 1)(x - 4)}{2}\right) + 11 \cdot \frac{(x - 1)(x - 2)}{6}\)

  \(P(x) = \frac{4(x - 2)(x - 4)- 9(x - 1)(x - 4) + 11(x - 1)(x - 2)}{6}\)

  \(P(x) = \frac{4x^2 - 24x + 32 - 9x^2 + 45x - 36 + 11x^2 - 33x + 22}{6}\)

  \(P(x) = \frac{6x^2 - 12x + 18}{6}\)

  \(P(x) = x^2 - 2x + 3\)

  Untuk mengestimasi nilai \(Y\) ketika \(X = 3\):

  \(P(3) = 3^2 - 2(3) + 3\)

  \(P(3) = 9 - 6 + 3\)

  \(P(3) = 6\)

\end{enumerate}
\end{document}
