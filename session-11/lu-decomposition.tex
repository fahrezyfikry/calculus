\documentclass{article}
\usepackage{amsmath}
\usepackage{amssymb}
\usepackage{geometry}

\geometry{
    paperwidth=170mm,
    paperheight=240mm,
    left=40pt,
    top=40pt,
    textwidth=280pt,
    marginparsep=20pt,
    marginparwidth=100pt,
    textheight=600pt,
    footskip=40pt
}
\begin{document}
\pagestyle{empty}

Diketahui:

\[
A = 
\begin{pmatrix}
2 & 1 & -2 \\
4 & -1 & -1 \\
2 & -1 & 1
\end{pmatrix}
\]

Dengan bentuk umumnya adalah \(A = LU\), dimana:

\[
A = LU, \quad
L =
\begin{pmatrix}
1 & 0 & 0 \\
l_{21} & 1 & 0 \\
l_{31} & l_{32} & 1
\end{pmatrix},
\quad
U =
\begin{pmatrix}
u_{11} & u_{12} & u_{13} \\
0 & u_{22} & u_{23} \\
0 & 0 & u_{33}
\end{pmatrix}
\]

\begin{enumerate}
    \item Menentukan \(u_{11}\), \(u_{12}\), dan \(u_{13}\) sebagai dasar dari \(U\):
  
    \(u_{11} = a_{11} = 2\)

    \(u_{12} = a_{12} = 1\)

    \(u_{13} = a_{13} = -2\)


    \item Menentukan \(l_{21}\) adan \(l_{31}\):

    \(l_{21} = \frac{a_{21}}{u_{11}} = \frac{4}{2} = 2\)

    \(l_{31} = \frac{a_{31}}{u_{11}} = \frac{2}{2} = 1\)

    \item Menentukan \(u_{22}\) dan \(u_{23}\) pada baris ke-2:
    
    \(B_2' = B_2 - l_{21}B_1\)

    \(B_2' = (4, -1, -1) - 2(2, 1, -2) \)

    \(B_2' = (4 - 4, -1 - 2, -1 + 4)\)
    
    \(B_2' = (0, -3, 3)\)

    maka, \(u_{22} = -3\) dan \(u_{23} = 3\)

    \item Menentukan \(l_{32}\):
    
    \(B_3' = B_3 - l_{31}B_1\)

    \(B_3' = (2, -1, 1) - 1(2, 1, -2) \)

    \(B_3' = (2 - 2, -1 - 1, 1 + 2)\)

    \(B_3' = (0, -2, 3)\)

    maka, \(l_{32} = \frac{a_{32}'}{u_{22}} = \frac{-2}{-3} = \frac{2}{3}\)

    \item Menentukan \(u_{33}\):

    \(u_{33} = a_{33}' - l_{32}u_{23}\)

    \(u_{33} = 3 - \frac{2}{3}(3) = 3 - 2 = 1\)

    \item Sehingga diperoleh matriks \(L\) dan \(U\):

    \[
    L =
    \begin{pmatrix}
    1 & 0 & 0 \\
    2 & 1 & 0 \\
    1 & \frac{2}{3} & 1
    \end{pmatrix},
    \quad
    U =
    \begin{pmatrix}
    2 & 1 & -2 \\
    0 & -3 & 3 \\
    0 & 0 & 1
    \end{pmatrix}
    \]
\end{enumerate}
\end{document}
