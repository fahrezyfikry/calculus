\documentclass[a4paper]{article}
\usepackage[margin=1.5cm]{geometry}
\usepackage{amsmath}
\usepackage{amssymb}
\usepackage{enumitem}
\begin{document}
\pagestyle{empty}

\begin{center}
    {\LARGE \textbf{Tugas Personel ke-2}}\\[0.5em]
    {\Large \textbf{Week 7}}
\end{center}

\begin{enumerate}[leftmargin=*]
  \item Cari semua turunan parsial kedua dari persamaan Berikut (LO1, max point = 20)
  \begin{enumerate}
    \item \(f(x,y) = x^4y - 2x^3y^2\)
    \item \(z = \frac{y}{2x+3y}\)
  \end{enumerate}
  \item Jelaskan F dan buat sketsa beberapa vektor dalam medan vektor F(x,y) yang diberikan oleh (LO1, max point = 20)
  \begin{enumerate}
    \item \(f(x,y) = xi + \frac{1}{2}yj\)
    \item \(f(x,y) = yi + (x+y)j\)
  \end{enumerate}
  \item Gunakan Crammar Rule untuk menyelesaikan (LO2, max point = 30)

  \begin{flalign*}
  &2x_1 + x_2 + x_3 = 2 &&\\
  &5x_1 + 2x_2 - 2x_3 = 9 &&\\
  &3x_1 - x_2 + x_3 = 5 &&
  \end{flalign*}

  \item Gunakan Gauss-Jordan untuk menyelesaikan (LO2, max point = 30)
  \begin{flalign*}
  &2x_1 + x_2 + x_3 = 2 &&\\
  &5x_1 + 2x_2 - 2x_3 = 9 &&\\
  &3x_1 - x_2 + x_3 = 5 &&
  \end{flalign*}
\end{enumerate}

\end{document}
