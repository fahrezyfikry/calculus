\documentclass{article}
\usepackage{amsmath}
\usepackage{amssymb}
\begin{document}
\pagestyle{empty}

Diketahui \textbf{sistem persamaan linear}:
\begin{align*}
2x + y + z &= 9 \\
x + 2y - z &= 6 \\
3x - y + 2z &= 17
\end{align*}

\begin{enumerate}
    \item Perubahan dari sistem persamaan linear ke \textbf{Augmented Matrix}:
    \[
    \begin{bmatrix}
    2 & 1 & 1 & \vert & 9 \\
    1 & 2 & -1 & \vert & 6 \\
    3 & -1 & 2 & \vert & 17
    \end{bmatrix}
    \]
    \item Menukar dua baris (\(R_i \leftrightarrow R_j\)) dengan menggunakan \textbf{Operasi Baris Elementer (OBE)}:
    
    Untuk kolom pertama, terdapat angka 1 pada baris ke-2, sehingga kita tukar baris pertama dengan baris kedua \(R_1 \leftrightarrow R_2\), untuk mempermudah perhitungan.

    \[
    \begin{bmatrix}
    1 & 2 & -1 & \vert & 6 \\
    2 & 1 & 1 & \vert & 9 \\
    3 & -1 & 2 & \vert & 17
    \end{bmatrix}
    \]

    \item Eliminasi $x$ pada baris 2 dan 3.

    Baris 2: \(B_2 = B_2 - 2B_1\)

    \(B_2 = [2, 1, 1 \vert 9] - 2[1, 2, -1 \vert 6]\)

    \(B_2 = [0, -3, 3 \vert -3]\)

    Baris 3: \(B_3 = B_3 - 3B_1\)

    \(B_3 = [3, -1, 2 \vert 17] - 3[1, 2, -1 \vert 6]\)

    \(B_3 = [0, -7, 5 \vert -1]\)

    \vspace{1em}

    Sehingga matriks menjadi:
    \[
    \begin{bmatrix}
    1 & 2 & -1 & \vert & 6 \\
    0 & -3 & 3 & \vert & -3 \\
    0 & -7 & 5 & \vert & -1
    \end{bmatrix}
    \]

    \item Ubah baris 2 agar leading 1, \(B_2 \div (-3)\), \(B_2 = \frac{1}{-3}B_2\)

    \(B_2 = - \frac{1}{3}B_2\)

    \[
    \begin{bmatrix}
    1 & 2 & -1 & \vert & 6 \\
    0 & 1 & -1 & \vert & 1 \\
    0 & -7 & 5 & \vert & -1
    \end{bmatrix}
    \]

    \item Eliminasi $y$ pada baris 1 dan 3.

    Baris 1: \(B_1 = B_1 - 2B_2\)

    \(B_1 = [1, 2, -1 \vert 6] - 2[0, 1, -1 \vert 1]\)

    \(B_1 = [1, 0, 1 \vert 4]\)

    Baris 3: \(B_3 = B_3 + 7B_2\)

    \(B_3 = [0, -7, 5 \vert -1] + 7[0, 1, -1 \vert 1]\)

    \(B_3 = [0, 0, -2 \vert 6]\)

    \vspace{1em}

    Sehingga matriks menjadi:
    \[
    \begin{bmatrix}
    1 & 0 & 1 & \vert & 4 \\
    0 & 1 & -1 & \vert & 1 \\
    0 & 0 & -2 & \vert & 6
    \end{bmatrix}
    \]

    \item Ubah baris 3 agar leading 1, \(B_3 \div (-2)\), \(B_3 = \frac{1}{-2}B_3\)

    \(B_3 = - \frac{1}{2}B_3\)

    \[
    \begin{bmatrix}
    1 & 0 & 1 & \vert & 4 \\
    0 & 1 & -1 & \vert & 1 \\
    0 & 0 & 1 & \vert & -3
    \end{bmatrix}
    \]

    \item Eliminasi $z$ pada baris 1 dan 2.
    
    Baris 1: \(B_1 = B_1 - B_3\)

    \(B_1 = [1, 0, 1 \vert 4] - 1[0, 0, 1 \vert -3]\)

    \(B_1 = [1, 0, 0 \vert 7]\)

    Baris 2: \(B_2 = B_2 + B_3\)

    \(B_2 = [0, 1, -1 \vert 1] + 1[0, 0, 1 \vert -3]\)

    \(B_2 = [0, 1, 0 \vert -2]\)

    \vspace{1em}

    Sehingga matriks menjadi:
    \[
    \begin{bmatrix}
    1 & 0 & 0 & \vert & 7 \\
    0 & 1 & 0 & \vert & -2 \\
    0 & 0 & 1 & \vert & -3
    \end{bmatrix}
    \]

\end{enumerate}

Sehingga, diperoleh nilai-nilai variabelnya adalah:

\[
x = 7, \quad y = -2, \quad z = -3
\]

\end{document}
