\documentclass{article}
\usepackage{amsmath}
\usepackage{amssymb}
\begin{document}
\pagestyle{empty}

\begin{tabular}{@{}l l@{}}
Nama & : Fikry Fahrezy Ramadhan \\
NIM & : 2802658263 \\
\end{tabular}

\vspace{1em}

Diketahui sistem persamaan linear berikut:

\[
\left\{
\begin{array}{rcl}
2x + y - 2z &=& -1 \\
3x - 3y - z &=& 5 \\
x - 2y + 3z &=& 6
\end{array}
\right.
\]

\begin{enumerate}
    \item Perubahan dari sistem persamaan linear ke \textbf{Augmented Matrix}:
    \[
    \begin{bmatrix}
    2 & 1 & -2 & \vert & -1 \\
    3 & -3 & -1 & \vert & 5 \\
    1 & -2 & 3 & \vert & 6
    \end{bmatrix}
    \]
    \item Merubah ke bentuk \textbf{eselon baris} dengan menggunakan \textbf{Operasi Baris Elementer (OBE)}:

    \begin{enumerate}
        \item Menukar dua baris (\(R_i \leftrightarrow R_j\))
        
        \vspace{1em}

        Pada kolom pertama, terdapat angka 1 pada baris ke-3, sehingga kita tukar baris pertama dengan baris ketiga \(R_1 \leftrightarrow R_3\), untuk mempermudah perhitungan.

        Dengan aturan \textbf{menukar baris}, sehingga matrix-nya menjadi:

        \[
        \begin{bmatrix}
        1 & -2 & 3 & \vert & 6 \\
        3 & -3 & -1 & \vert & 5 \\
        2 & 1 & -2 & \vert & -1
        \end{bmatrix}
        \]

        \vspace{1em}

        \item Mengalikan suatu baris dengan konstanta bukan nol (\(kR_i\), \(k \neq 0\))
        
        \begin{enumerate}
            \item Buat elemen-elemen di bawah pivot (elemen pertama pada baris pertama) menjadi \textbf{nol}.

        \end{enumerate}

        \vspace{1em}

        Untuk \textbf{baris ke-2} dan \textbf{baris ke-3}, gunakan \textbf{rumus umum eliminasi}:

        \vspace{1em}

        Pada baris ke-2, elemen yang ingin dihilangkan adalah \( a_{21} = 3 \).

        \(R_2 = R_2 - 3R_1\)

        \(R_2 = [3, -3, -1 \vert 5] - 3[1, -2, 3 \vert 6]\)

        \(R_2 = [0, 3, -10 \vert -13]\)

        \vspace{1em}
        
        Pada baris ke-3, elemen yang ingin dihilangkan adalah \( a_{31} = 2 \).

        \(R_3 = R_3 - 2R_1\)

        \(R_3 = [2, 1, -2 \vert -1] - 2[1, -2, 3 \vert 6]\)

        \(R_3 = [0, 5, -8 \vert -13]\)

        Sehingga hasilnya:
        \[
        \begin{bmatrix}
        1 & -2 & 3 & \vert & 6 \\
        0 & 3 & -10 & \vert & -13 \\
        0 & 5 & -8 & \vert & -13
        \end{bmatrix}
        \]

        \vspace{1em}
    \end{enumerate}
\end{enumerate}
\end{document}
